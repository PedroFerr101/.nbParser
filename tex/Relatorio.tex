\documentclass[a4paper]{article}

\usepackage[utf8]{inputenc}
\usepackage[T1]{fontenc} 
\usepackage[portuges]{babel}
\usepackage{a4wide}
\usepackage{graphicx}
\usepackage{listings}
\usepackage{caption}
\usepackage{subcaption}
\usepackage{hyperref}
\usepackage{xcolor}
\usepackage{multicol}
\usepackage{listingsutf8}


\title{Projeto de Sistemas Operativos\\Grupo 20}
\author{Henrique Pereira (a80261) \and Pedro Moreira (a82364) \and Pedro Ferreira (a81135) }



%% DEFINIÇÃO DOS SNIPPETS EM C
\usepackage{listings}
\lstdefinestyle{customc}{
  belowcaptionskip=1\baselineskip,
  breaklines=true,
  language=C,
  showstringspaces=false,
  basicstyle=\footnotesize\ttfamily,
  keywordstyle=\bfseries\color{green!40!black},
  morekeywords={Comando},keywordstyle=\bfseries\color{green!40!black},
  commentstyle=\itshape\color{gray},
  stringstyle=\color{orange},
}

\lstset{escapechar=@,style=customc}




\begin{document}

\maketitle

\newpage

\tableofcontents

\newpage

\section{Introdução}
\label{sec:intro}

No contexto da Unidade Curricular Sistemas Operativos (SO), foi-nos proposto o desenvolvimento de um processador de \textit{Notebooks} na linguagem \textit{C}. 
Um \textit{Notebook} é um ficheiro de texto que mistura fragmentos de código, resultados de execução e documentação. 


Na secção \ref{sec:note} serão apresentadas as estratégias criadas pelo grupo para a concretização dos requisitos apresentados.
Nas subsecções \ref{sec:funcionalidades} e \ref{sec:testes}, serão, respetivamente, descritas pequenas funcionalidades do programa e 
apresentados \textit{notebooks} como exemplos e seus respetivos outputs.
Por fim, na secção \ref{sec:reflexao} será feita uma reflexão crítica sobre o desempenho da nossa solução e será discutida uma outra.


%%%%%%%%%%%%%%%%%%%%%%%%%%%%%%%%%%%%%%%%%%%%%
\section{Processador de Notebook}
\label{sec:note}

Como já referido, um \textit{Notebook} é uma mistura de vários tipos de texto. Perante o enunciado proposto e, desde pronto, a pensar 
nas \textit{Funcionalidades Avançadas} o grupo decidiu criar uma estrutura que guardasse os fragmentos de código (comandos) e todas as 
dependências entre si.
Essa estrutura tem a seguinte definição: 

\begin{lstlisting}[caption=Definição da estrutura]
typedef struct cmd{
    char* text;
    char** cmd;
    char* full_cmd; 
    int* needs_me;
    int needs_me_len;
    int my_input_id;
}*Comando;
\end{lstlisting}

\begin{itemize}
  \item{\texttt{text} - corresponde à documentação do respetivo fragmento de código}
  \item{\texttt{cmd} - fragmento de código a ser executado}
  \item{\texttt{full\_cmd} - linha completa com comando}
  \item{\texttt{needs\_me} - índices dos comandos que necessitam do output gerado por este comando}
  \item{\texttt{needs\_me\_len} - número de elementos do vetor \texttt{needs\_me}} 
  \item{\texttt{my\_input\_id} - índice do input necessário para a execução deste comando}
\end{itemize}


%%%%%%%%%%%%%%%%%%%%%%%%%%%%%%%%%%%%%%%%%%%%%
\subsection{Estratégia}

%% Fase inicial
A nossa estratégia começa por carregar o ficheiro todo para memória, por razões de eficiência, em vez de estar a realizar várias 
leituras do mesmo, e para facilitar o seu manuseamento. Após o seu carregamento completo para um buffer, fazemos o parser para a nossa 
estrutura de dados. 


Esta estrutura permite, de forma simples, a execução dos comandos contidos no ficheiro (através da systemcall \texttt{execvp}),
assim como a reconstrução do ficheiro original sem qualquer perda de dados. O parser é feito com o auxílio de várias funções como, por 
exemplo, a função \texttt {getDependentNumber} que, como requisito das \textit{Funcionalidades Avançadas}, atribui ao campo da 
estrutura \texttt{my\_input\_id} o ID relativo do output necessário para a execução do comando.

%% Fase de execução
BLABLABLA

%% Fase de escrita
Após o carregamento e a execução concorrente dos comandos presentes no \textit{Notebook} passado como argumento, chega a fase de juntar os resultados de todos estes comandos ao \textit{Notebook}. Para tal, bastou-nos aceder a cada um dos ficheiros auxiliares e escrevê-los (com o \texttt{write}) num novo ficheiro \textit{.nb}, encadeando estes resultados com o texto já existente no ficheiro-argumento original, através dos delimitadores \texttt{>>>} e \texttt{<<<}. 

Após este processo, com o novo \textit{Notebook} já pronto, foi apenas necessário substituir o ficheiro antigo por este, utilizando para isso a \textit{system call} \texttt{rename}. Para efeitos de limpeza, apagamos, de seguida, todos os ficheiros auxiliares até agora criados (exceto o \textit{.nb}), assim como a pasta que os continha, cujo \textit{path} é o seguinte: \texttt{/tmp/notebook}.


%%%%%%%%%%%%%%%%%%%%%%%%%%%%%%%%%%%%%%%%%%%%%
\subsection{Funcionalidades}
\label{sec:funcionalidades}

O utilizador pode necessitar de um output que não é o que o comando anterior devolveu. Para isso o utilizador pode escrever 
\texttt{\$x|}, em que \textit{x} é o número do output do \textit{x-ésimo} comando anterior. 

Para que não existam demasiadas \textit{pré-condições} aquando da escrita do \textit{notebook}, decidimos que a documentação do 
comando não necessitava de ser apenas uma linha de texto.
Estas funcionalidades serão demonstradas na secção \ref{sec:testes}.
O utilizador pode interromper o programa a qualquer momento através da combinação \texttt{Ctrl + C}. Esta interrupção faz com que o 
ficheiro não seja alterado.

Para além disso, caso haja um erro, o programa para a sua execução e não altera o ficheiro.




%%%%%%%%%%%%%%%%%%%%%%%%%%%%%%%%%%%%%%%%%%%%%
\subsection{Testes}
\label{sec:testes}

%% Exemplo 1
\begin{lstlisting}[caption=1º Exemplo]
Este comando lista os ficheiros:
$ ls
Agora podemos ordenar estes ficheiros:
$| sort
E escolher o primeiro:
$| head -1
\end{lstlisting}

%% Resultado 1
\begin{lstlisting}[caption=Resultado do 1º Exemplo]

\end{lstlisting}


%% Exemplo 2
\begin{lstlisting}[caption=2º Exemplo]
Este comando imprime a diretoria corrente:
$ pwd
E podemos contar o tamanho em palavras:
$| wc -w
Ou podemos contar o tamanho em carateres:
$2| wc -c
\end{lstlisting}

%% Resultado 2

\begin{lstlisting}[caption=Resultado do 2º Exemplo]

\end{lstlisting}


%% Exemplo 3
\begin{lstlisting}[caption=3º Exemplo]
Bom dia,
Apresento o meu exemplo de notebook:
Para listar todo o conteudo duma diretoria:
$ ls -a
\end{lstlisting}

%% Resultado 3

\begin{lstlisting}[caption=Resultado do 3º Exemplo]

\end{lstlisting}




%%%%%%%%%%%%%%%%%%%%%%%%%%%%%%%%%%%%%%%%%%%%%
\section{Reflexão}
\label{sec:reflexao}

A estratégia que criamos para a resolução do problema tem os seguintes problemas:

\begin{itemize}
  \item{Uso desnecessário de memória - asdasdfasdf}
  \item{Possibilidade de bloqueio - asdasdasda}
\end{itemize}

O grupo também tem a noção de que a solução não é a ideal .............


%%%%%%%%%%%%%%%%%%%%%%%%%%%%%%%%%%%%%%%%%%%%%
\section{Conclusões}
\label{sec:conclusao}

Este projeto foi extremamente enriquecedor para o grupo, uma vez que nos permitiu pôr em prática os conhecimentos adquiridos nas aulas de SO.

Em suma, na nossa opinião, os requisitos propostos foram cumpridos e conjugados com êxito, sendo que o resultado final, apesar de não ser o ideal, revela-se apelativo e funcional.

Foi, no geral, um trabalho  altamente pedagógico, que nos permitiu desenvolver não só competências essenciais ao curso de Mestrado Integrado em Engenharia Informática, como também competências sociais, tais como o trabalho em grupo.

\end{document}\grid
