\documentclass[a4paper]{article}

\usepackage[utf8]{inputenc}
\usepackage[portuges]{babel}
\usepackage{a4wide}
\usepackage{graphicx}
\usepackage{listings}
\usepackage{caption}
\usepackage{subcaption}
\title{Projeto de Sistemas Operativos\\Grupo 20}
\author{Henrique Pereira (a80261) \and Pedro Moreira (a82364) \and Pedro Ferreira (a81135) }



%% DEFINIÇÃO DOS SNIPPETS EM C
\usepackage{listings}
\lstdefinestyle{customc}{
  belowcaptionskip=1\baselineskip,
  breaklines=true,
  language=C,
  showstringspaces=false,
  basicstyle=\footnotesize\ttfamily,
  keywordstyle=\bfseries\color{green!40!black},
  morekeywords={GArray,GHashTable,GTree,STR_pair,LONG_list,LONG_pair,USER,TCD_community, User,Post, Tag},keywordstyle=\bfseries\color{green!40!black},
  commentstyle=\itshape\color{gray},
  stringstyle=\color{orange},
}

\lstset{escapechar=@,style=customc}




\begin{document}

\maketitle

\newpage

\tableofcontents

\newpage

\section{Introdução}
\label{sec:intro}

No contexto da Unidade Curricular Sistemas Operativos (SO), foi-nos proposto o desenvolvimento de um processador de \textit{Notebooks}. 
Um \textit{Notebook} é um ficheiro de texto que mistura fragmentos de código, resultados de execução e documentação. 
Este projeto foi desenvolvido na linguagem \textit{C}.
Na secção \ref{sec:note} serão apresentadas as estratégias criadas pelo grupo para a concretização dos requisitos apresentados.
Na secção \ref{sec:reflexao} será feita uma reflexão crítica sobre o desempenho da nossa solução e será discutida uma outra.


%%%%%%%%%%%%%%%%%%%%%%%%%%%%%%%%%%%%%%%%%%%%%
\section{Processador de Notebook}
\label{sec:note}

Como já referido, um \textit{Notebook} é uma mistura de vários tipos de texto. Perante o enunciado proposto e, desde pronto, a pensar nas 
\textit{Funcionalidades Avançadas} o grupo decidiu criar uma estrutura que guardasse os fragmentos de código (comandos) e todas as dependências entre si.
Essa estrutura tem os seguintes campos: 

%%%% CODIGO BONITINHO DA ESTRUTURA

\begin{itemize}
  \item{text - corresponde à documentação do respetivo fragmento de código}
  \item{cmd - fragmento de código a ser executado}
  \item{needsme - índices dos comandos que necessitam do output gerado por este comando} %% NÃO DÁ COM _ NO NOME
  \item{needsmelen - } %% É PRECISO ESTE?
  \item{myinputid - índice do input necessário para a execução deste comando}
\end{itemize}

%% Falar que a estratégia passa pelo carregamento do ficheiro para um buffer, parser para a estrutura bla bla bla...	



%%%%%%%%%%%%%%%%%%%%%%%%%%%%%%%%%%%%%%%%%%%%%
\section{Reflexão}
\label{sec:reflexao}

A estratégia que criamos para a resolução do problema tem os seguintes problemas:

\begin{itemize}
  \item{Uso desnecessário de memória - asdasdfasdf}
  \item{Possibilidade de bloqueio - asdasdasda}
\end{itemize}

O grupo também tem a noção de que a solução não é a ideal .............


%%%%%%%%%%%%%%%%%%%%%%%%%%%%%%%%%%%%%%%%%%%%%
\section{Conclusões}
\label{sec:conclusao}

Este projeto foi extremamente enriquecedor para o grupo, uma vez que nos permitiu pôr em prática os conhecimentos adquiridos nas aulas de SO.

Em suma, na nossa opinião, os requisitos propostos foram cumpridos e conjugados com êxito, sendo que o resultado final, apesar de não ser o ideal, revela-se apelativo e funcional.

Foi, no geral, um trabalho  altamente pedagógico, que nos permitiu desenvolver não só competências essenciais ao curso de Mestrado Integrado em Engenharia Informática, como também competências sociais, tais como o trabalho em grupo.

\end{document}\grid
